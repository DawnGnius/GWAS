\section{Summary}\label{sec:summary}

% \section{Biological Interpretation}

Our results clearly demonstrate the sparsity and low-rankness of associations between eQTLs and genes, which allows us to conclude that a certain linear combination of eQTLs has effects only on a subset of genes. Thus, to study a certain group of genes, a reasonable and efficient way is to focus on the corresponding  set of eQTLs.

The SVD layers provide more information on samples and genes. The plot for layer 1 indicates that the yeast samples can be divided into two clusters, which means our method can do classification based on the latent variables. As to genes, the nonzero entries in each column of $V$ match with dominating genes in each layer. The first layer is dominated by four genes, including STE3(-0.69), STE2(0.61), MAT$\alpha$1(-0.32) and MAT$\alpha$2(-0.17). All four genes are upstream in the pheromones pathway, as shown in the top right corner of Figure~\ref{fig:MAPK}. The second layer include the leading genes CTT1(-0.94), GLO1(-0.14), SLN1(0.14), SLT2(-0.13), MSN4(-0.12) and STE2(-0.11). Notably, MSN4, GLO1, and CTT1 are all downstream genes linked to the downstream gene SLN1 in the high osmolarity pathway. The leading genes in the third layer can be divided into two groups. The upstream group has STE2(0.26), GPA1(0.20), WSC3(0.19) STE3(0.18) and SLN1(0.15). The downstream group includes FUS1(0.81), FAR1(0.29), STE12(0.11) and TEC1(0.11). They are mainly located in the pheromone and the starvation pathway. Furthermore, STE2 is a leading gene in all three layers. Above all, our results suggest a strong linkage between the upstream and downstream genes, as well as three or four  functionally distinct patterns in the MAPK pathway, where each pattern is regulated by a subset of genes.  This conclusion is consistent with current findings of genetical studies.
