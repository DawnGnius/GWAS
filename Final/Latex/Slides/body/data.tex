% ----------------------------------------------------------------------------------------
% 	PAGE Section page
% ----------------------------------------------------------------------------------------
\section{Yeast Gene Expression Data}
\begin{frame}
    \sectionpage 
    \footnote[1]{The yeast data can be accessed in Gene Expression Omnibus(GEO) by accession number GSE1990.}
    \footnote[2]{The data were derived from a cross between two strains of the budding yeast: BY4716 and RM11-1a.}
    \footnote[3]{Brem, R. B., Kruglyak, L. (2005). The landscape of genetic complexity across 5,700 gene expression traits in yeast. Proceedings of the National Academy of Sciences, 102(5), 1572-1577.}
\end{frame}

<<<<<<< HEAD
\begin{frame}\frametitle{Yeast Data}
    The yeast data can be accessed in Gene Expression Omnibus(GEO) by accession number GSE1990. 
    The data were derived from a cross between two strains of the budding yeast: BY4716 and RM11-1a \footnote[1]{Brem, R. B., Kruglyak, L. (2005). The landscape of genetic complexity across 5,700 gene expression traits in yeast. Proceedings of the National Academy of Sciences, 102(5), 1572-1577.}. 

%    Gene expression measurements were obtained for $6216$ open reading frames in $112$ segregants, and genotypes were identified at $3244$ markers. 

\end{frame}

%\begin{frame}
%    \begin{table}[h]
%        \centering
%        \begin{tabular}{|l|p{7cm}|}
%            \hline
%            Title                           &   Genetic complexity in yeast transcripts \\ \hline
%            Organism                        &   Saccharomyces cerevisiae                \\ \hline
%            Experiment type                 &   Expression profiling by array           \\ \hline
%            \multirow{4}{*}{Data Size}      &   Data set consists of a $3244\times112$ \textbf{genotype matrix} with $3244$ genotypes in rows and $112$ samples in columns and a $6216\times112$ \textbf{gene expression matrix} with $6216$ genes in rows and $112$ samples in columns.  \\ \hline
%            \multirow{3}{*}{Description}    &   cDNA array of Saccharomyces cerevisiae ORFs. Genotype is category variable, and gene expression level is given by  $log_2(\text{sample} / \text{BY reference})$\\ 
%            \hline
%        \end{tabular}
%        \caption{Information About Data}
%    \end{table}
%
%\end{frame}


\begin{frame}\frametitle{Yeast Data}
    \begin{enumerate}
        \item Why Yeast?
        \item What does our data describe?
        \item How is the data measured? 
    \end{enumerate}
\end{frame}

\begin{frame}\frametitle{Why yeast?}
    \begin{itemize}
    	\item complete genome sequence  
		\begin{itemize}
			\item no missing data 
		\end{itemize}    	
		\item share some genes with human cells   
		\begin{itemize}
			\item study human diseases
			\item test new drugs
		\end{itemize}		     
        %\item single-celled eukaryotes with short life cycle

        
    \end{itemize}
\end{frame}

=======
% ----------------------------------------------------------------------------------------
% 	PAGE Brief introduction to the data.
% ----------------------------------------------------------------------------------------
\begin{frame}\frametitle{Why Yeast?}
    \begin{enumerate}
        \item share many genes similar with human cells
        
        \item single-celled eukaryotes with short life cycle
        
        \item the growth can be controlled easily by experimenters
    \end{enumerate}
\end{frame}

% ----------------------------------------------------------------------------------------
% 	PAGE Data structure
% ----------------------------------------------------------------------------------------
>>>>>>> 022b9d4b989cf4e0183b121d8dd02068dc8ddbfc
\begin{frame}{Data Description}
\begin{table}[h]
        \centering

        \begin{tabular}{|l|p{7cm}|}
            \hline
            Title                           &   Genetic complexity in yeast transcripts \\ \hline
            Organism                        &   Saccharomyces cerevisiae (or Baker's yeast)                \\ \hline
            Experiment type                 &   Expression profiling by array           \\ \hline
<<<<<<< HEAD
            \multirow{4}{*}{Data Size}      &   Data set consists of a $3244\times112$ \textbf{genotype matrix} with $3244$ genotypes in rows and $112$ samples in columns and a $6216\times112$ \textbf{gene expression matrix} with $6216$ genes in rows and $112$ samples in columns.  \\ \hline
%            \multirow{3}{*}{Description}    &   cDNA array of Saccharomyces cerevisiae ORFs. Genotype is category variable, and gene expression level is given by  $log_2(\text{sample} / \text{BY reference})$\\ 
            %\hline
=======
            \multirow{3}{*}{Data Size}      &   112 yeast samples. Data set consists of $3244$ genotypes and $6216$ genes.  $X \in \mathbb{R}^{3244\times112}$, $Y \in \mathbb{R}^{6216\times112}$. \\ \hline
           \multirow{3}{*}{Description}     &   cDNA array of saccharomyces cerevisiae ORFs. Genotype is category variable, and gene expression level is given by  $log_2(\text{sample} / \text{BY reference})$. \\ \hline
>>>>>>> 022b9d4b989cf4e0183b121d8dd02068dc8ddbfc
        \end{tabular}
        \caption{Information About Data}
    \end{table}
\end{frame}

<<<<<<< HEAD
\begin{frame}\frametitle{Structure of data}
    \begin{itemize}
        \item the value of the eQTLs: 1 or 2
    
        (still need some explanations here)
        % we add 1 to each original value for the convenient for mathematical processing
=======
% \begin{frame}\frametitle{Structure of data}
%     \begin{itemize}
%         \item the value of the eQTLs: 1 or 2
    
%         (still need some explanations here)
%         % we add 1 to each original value for the convenient for mathematical processing
>>>>>>> 022b9d4b989cf4e0183b121d8dd02068dc8ddbfc

%         \item Y: almost continuous because of the use of microarray technology (microarray) to sequence the samples.
      
%     \end{itemize}
% \end{frame}

% ----------------------------------------------------------------------------------------
% 	PAGE Other informations ????
% ----------------------------------------------------------------------------------------
\begin{frame}\frametitle{Structre of Data}
    \begin{itemize}
%        \item analyze the influence of eQTLs(the quantitative trait loci) on the expression level of genes in the yeast MAPK signaling pathways. 
%        \item Biological characteristics of variables in the study(Gustin et al.(1998), and Brem and Kruglyak(2005)):
        \item a few functionally distinct signaling pathways of genes exist
        \item the association structure between the eQTLs and the gene is of low rank
        \item each signaling pathway involves only a subset of genes, which are regulated by only a few genetic variants
    \end{itemize}
\end{frame}
   

<<<<<<< HEAD
\begin{frame}\frametitle{Structre of Data}
    \begin{itemize}
%        \item analyze the influence of eQTLs(the quantitative trait loci) on the expression level of genes in the yeast MAPK signaling pathways. 
%        \item Biological characteristics of variables in the study(Gustin et al.(1998), and Brem and Kruglyak(2005)):
        \item a few functionally distinct signaling pathways of genes exist
        \item the association structure between the eQTLs and the gene is of low rank
        \item each signaling pathway involves only a subset of genes, which are regulated by only a few genetic variants
    \end{itemize}
\end{frame}



   

\begin{frame}\frametitle{Question of Interesting}
    \begin{enumerate}
        % \renewcommand{\labelenumi}{(\theenumi)}
        \item How the yeast eQTLs(expression quantitative trait loci), which are regions of the genome containing DNA sequence variants, influence the expression level of genes? 
        \item What is the influence of eQTLs on the genes involved in the yeast MAPK signaling pathways?
    \end{enumerate}
\end{frame}

\begin{frame}\frametitle{ Questions}
=======

% ----------------------------------------------------------------------------------------
% 	PAGE what we want to do in Statistics??
%   As is described before in Introduction Section. Our goal is to link genetic variations to gene expression levels. Here we repeat again, and give a equivalent statistics form.
%   the second block confuses me. eQTL?? Genes?
% ----------------------------------------------------------------------------------------

% \begin{frame}\frametitle{Question of Interesting}
%     \begin{enumerate}
%         % \renewcommand{\labelenumi}{(\theenumi)}
%         \item How the yeast eQTLs(expression quantitative trait loci), which are regions of the genome containing DNA sequence variants, influence the expression level of genes? 
%         \item What is the influence of eQTLs on the genes involved in the yeast MAPK signaling pathways?
%     \end{enumerate}
% \end{frame}

\begin{frame}\frametitle{Question of Interesting}
>>>>>>> 022b9d4b989cf4e0183b121d8dd02068dc8ddbfc
    \begin{block}{Question of Interesting}
        \begin{itemize}
        \item How the eQTLs influence the expression level of genes in the yeast MAPK signaling pathways by analyzing a yeast dataset?
        \item Which group of eQTLs affect certain group of genes?
        \end{itemize}
    \end{block}
    
<<<<<<< HEAD
    \begin{block}{equivalent question in statistics}
        reveal multiple distinct associations between subsets of genes and subsets of genetic variants(eQTLs)
    \end{block}
\end{frame}


\section{Data Preparation}
\begin{frame}
    \sectionpage
\end{frame}

\begin{frame}
    \frametitle{Processing Markers Data}

    \begin{itemize}
        \item Hierarchical clustering by complete distance ...
        \item Select representative markers.
        \item Marginal gene-marker association analysis.
    \end{itemize}
\end{frame}

\begin{frame}
    \frametitle{Processing Expression Level Data}

    We choose genes according to MAPK signaling pathways \footnote[2]{Kanehisa, M., Goto, S., Sato, Y., Kawashima, M., Furumichi, M. and Tanabe, M. (2014) Data, information, knowledge and principle: Back to metabolism in KEGG. Nucleic Acids Res., 42, D199–D205.}

    * * Add pictures ?? 
\end{frame}
=======
    \begin{block}{Equivalent Question in Statistics}
        Reveal multiple distinct associations between subsets of genes and subsets of genetic variants(eQTLs).
    \end{block}
\end{frame}
>>>>>>> 022b9d4b989cf4e0183b121d8dd02068dc8ddbfc
