% ----------------------------------------------------------------------------------------
% 	PAGE Section page
% ----------------------------------------------------------------------------------------
\section{Yeast Gene Expression Data}
\begin{frame}
    \sectionpage 
    \footnotetext[1]{The yeast data can be accessed in Gene Expression Omnibus(GEO) by accession number GSE1990.}
    \footnotetext[2]{The data were derived from a cross between two strains of the budding yeast: BY4716 and RM11-1a.}
    \footnotetext[3]{Brem, R. B., Kruglyak, L. (2005). The landscape of genetic complexity across 5,700 gene expression traits in yeast. Proceedings of the National Academy of Sciences, 102(5), 1572-1577.}
\end{frame}

% ----------------------------------------------------------------------------------------
% 	PAGE Brief introduction to the data.
% ----------------------------------------------------------------------------------------
\begin{frame}\frametitle{Why yeast?}
    \begin{itemize}
    	\item Complete genome sequence  
   	
		\item Share some genes with human cells   
%		\begin{itemize}
%			\item study human diseases
%			\item test new drugs
%		\end{itemize}		     
        %\item single-celled eukaryotes with short life cycle

        
    \end{itemize}
\end{frame}

% ----------------------------------------------------------------------------------------
% 	PAGE Data structure
% ----------------------------------------------------------------------------------------
\begin{frame}{Data Description}
\begin{table}[h]
        \centering

        \begin{tabular}{|l|p{7cm}|}
            \hline
            Title                           &   Genetic complexity in yeast transcripts \\ \hline
            Organism                        &   Saccharomyces cerevisiae (Baker's yeast)\\ \hline
            Experiment type                 &   Expression profiling by array           \\ \hline
            \multirow{3}{*}{Data Size}      &   112 yeast samples. Data set consists of $3244$ genotypes and $6216$ genes.  $X \in \mathbb{R}^{3244\times112}$, $Y \in \mathbb{R}^{6216\times112}$. \\ \hline
           \multirow{3}{*}{Description}     &   Genotype\footnote[1]{eQTLs (expression Quantitative Trait Locis): some special SNPs which are associated with gene expression. } is a categorical variable, and gene expression level is given by  $log_2(\text{sample} / \text{BY reference})$. \\ \hline
        \end{tabular}
        \caption{Information About Data}
    \end{table}
\end{frame}

% \begin{frame}\frametitle{Structure of data}
%     \begin{itemize}
%         \item the value of the eQTLs: 1 or 2
    
%         (still need some explanations here)
%         % we add 1 to each original value for the convenient for mathematical processing

%         \item Y: almost continuous because of the use of microarray technology (microarray) to sequence the samples.
      
%     \end{itemize}
% \end{frame}

% ----------------------------------------------------------------------------------------
% 	PAGE Other informations ????
% ----------------------------------------------------------------------------------------
% \begin{frame}\frametitle{Structre of Data}
%     \begin{itemize}
% %        \item analyze the influence of eQTLs(the quantitative trait loci) on the expression level of genes in the yeast MAPK signaling pathways. 
% %        \item Biological characteristics of variables in the study(Gustin et al.(1998), and Brem and Kruglyak(2005)):
%         \item a few functionally distinct signaling pathways of genes exist
%         \item the association structure between the eQTLs and the gene is of low rank
%         \item each signaling pathway involves only a subset of genes, which are regulated by only a few genetic variants
%     \end{itemize}
% \end{frame}
   


% ----------------------------------------------------------------------------------------
% 	PAGE what we want to do in Statistics??
%   As is described before in Introduction Section. Our goal is to link genetic variations to gene expression levels. Here we repeat again, and give a equivalent statistics form.
%   the second block confuses me. eQTL?? Genes?
% ----------------------------------------------------------------------------------------

% \begin{frame}\frametitle{Question of Interesting}
%     \begin{enumerate}
%         % \renewcommand{\labelenumi}{(\theenumi)}
%         \item How the yeast eQTLs(expression quantitative trait loci), which are regions of the genome containing DNA sequence variants, influence the expression level of genes? 
%         \item What is the influence of eQTLs on the genes involved in the yeast MAPK signaling pathways?
%     \end{enumerate}
% \end{frame}

\begin{frame}\frametitle{Question of Interest}
    \begin{block}{Question of Interest}
        \begin{itemize}
        \item How eQTLs influence gene expression levels in the yeast MAPK signaling pathways ?
        \item Which group of eQTLs affect certain group of genes?
        \end{itemize}
    \end{block}
    
    \begin{block}{Equivalent Question in Statistics}
        Reveal multiple distinct associations between subsets of genes (eQTLs) and subsets of genetic variants. 
    \end{block}
\end{frame}
