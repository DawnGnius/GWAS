\section{Real Data: Yeast}
\begin{frame}
    \sectionpage
\end{frame}

\begin{frame}\frametitle{Data Description}
    \begin{enumerate}
        \item Why Yeast?
        \item Where is our data come from? 
        \item How does the data measured? 
        \item What does our data describe? 
    \end{enumerate}
\end{frame}

\begin{frame}
    \frametitle{Data Description}
    The data can be accessed in Gene Expression Omnibus(GEO) by accession number GSE1990. 
    The data were derived from a cross between two strains of the budding yeast: BY4716 and RM11-1a \footnote[1]{Brem, R. B., Kruglyak, L. (2005). The landscape of genetic complexity across 5,700 gene expression traits in yeast. Proceedings of the National Academy of Sciences, 102(5), 1572-1577.}. 

    Gene expression measurements were obtained for $6216$ open reading frames in $112$ segregants, and genotypes were identified at $3244$ markers. 

\end{frame}

\begin{frame}
    \begin{table}[h]
        \centering
        \begin{tabular}{|l|p{7cm}|}
            \hline
            Title                           &   Genetic complexity in yeast transcripts \\ \hline
            Organism                        &   Saccharomyces cerevisiae                \\ \hline
            Experiment type                 &   Expression profiling by array           \\ \hline
            \multirow{4}{*}{Data Size}      &   Data set consists of a $3244\times112$ \textbf{genotype matrix} with $3244$ genotypes in rows and $112$ samples in columns and a $6216\times112$ \textbf{gene expression matrix} with $6216$ genes in rows and $112$ samples in columns.  \\ \hline
            \multirow{3}{*}{Description}    &   cDNA array of Saccharomyces cerevisiae ORFs. Genotype is category variable, and gene expression level is given by  $log_2(\text{sample} / \text{BY reference})$\\ 
            \hline
        \end{tabular}
        \caption{Information About Data}
    \end{table}

\end{frame}

\begin{frame}\frametitle{Question of Interesting}
    \begin{enumerate}
        % \renewcommand{\labelenumi}{(\theenumi)}
        \item How the yeast eQTLs(expression quantitative trait loci), which are regions of the genome containing DNA sequence variants, influence the expression level of genes? 
        \item What is the influence of eQTLs on the genes involved in the yeast MAPK signaling pathways?
    \end{enumerate}
    

\end{frame}


\section{Data Preparation}
\begin{frame}
    \sectionpage
\end{frame}

\begin{frame}
    \frametitle{Processing Markers Data}

    \begin{itemize}
        \item Hierarchical clustering by complete distance ...
        \item Select representative markers.
        \item Marginal gene-marker association analysis.
    \end{itemize}
\end{frame}

\begin{frame}
    \frametitle{Processing Expression Level Data}

    We choose genes according to MAPK signaling pathways \footnote[2]{Kanehisa, M., Goto, S., Sato, Y., Kawashima, M., Furumichi, M. and Tanabe, M. (2014) Data, information, knowledge and principle: Back to metabolism in KEGG. Nucleic Acids Res., 42, D199–D205.}

    * * Add pictures ?? 
\end{frame}