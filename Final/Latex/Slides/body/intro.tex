\section{Introduction}
\begin{frame}
\sectionpage
\end{frame}

\begin{frame}
    \frametitle{Introduction}
<<<<<<< HEAD
    \begin{itemize}
      \item Background 
      \begin{equation*}
        \text{phenotype} \leftarrow 
        \text{gene expression level} \leftarrow 
        \text{genetic variation } 
=======
    \begin{block}{Background} 
      \begin{equation*}
        \text{phenotype} \leftarrow 
        \text{gene expression level} \leftarrow 
        \text{genetic variation }
>>>>>>> 022b9d4b989cf4e0183b121d8dd02068dc8ddbfc
      \end{equation*}
	\end{block}

%      \item Useful statistical analysis:
%      
%      Classification:genetic variants $\leftrightarrow$ gene expressions
	\begin{block}{Goal}
		Relation between genetic variations and gene expression levels.
	\end{block}

<<<<<<< HEAD
%      \item Useful statistical analysis:
%      
%      Classification:genetic variants $\leftrightarrow$ gene expressions
\item Goal: link genetic variation to gene expression level

    \end{itemize}
\end{frame}

%\begin{frame}
%	\frametitle{SNPs \& eQTLs}
%	\begin{itemize}
%		\item How to detect genetic variation ?
%		\begin{itemize}
%			\item genotype at SNPs
%		\end{itemize}
%		\item What are SNPs? (Fig. XXX
%		% figure
%		\begin{itemize}
%			\item E.g.SNPs occur every 100 to 300 bases in human genome.
%		\end{itemize}
%		
%	\end{itemize}
%\end{frame}
%
%\begin{frame}
%	\frametitle{SNPs \& eQTLs}
%	\begin{itemize}
%		\item eQTLs  (Fig. XXX
%		\begin{itemize}
%		\item  expression Quantitative Trait Locis
%		\item some special SNPs which are associated with gene expression 
%		\end{itemize}
%		%figure	
%	\end{itemize}
%\end{frame}
=======
    
\end{frame}

% \begin{frame}
% 	\frametitle{Details: SNPs \& eQTLs}
% 	\begin{itemize}
% 		\item How to detect genetic variation ?
% 		\begin{itemize}
% 			\item genotype at SNPs
% 		\end{itemize}
% 		\item What are SNPs? (Fig. XXX
% 		% figure
% 		\begin{itemize}
% 			\item E.g.SNPs occur every 100 to 300 bases in human genome.
% 		\end{itemize}
		
% 	\end{itemize}
% \end{frame}

% \begin{frame}
% 	\frametitle{Details: SNPs \& eQTLs cont.}
% 	\begin{itemize}
% 		\item eQTLs  (Fig. XXX
% 		\begin{itemize}
% 		\item  expression Quantitative Trait Locis
% 		\item some special SNPs which are associated with gene expression 
% 		\end{itemize}
% 		%figure	
% 	\end{itemize}
% \end{frame}

>>>>>>> 022b9d4b989cf4e0183b121d8dd02068dc8ddbfc
%\begin{frame}
%    \frametitle{ XX Methods of clustering (we can summary some methods??)}
%    \begin{itemize}
%      \item Question:
%   
%    Which clustering techniques are likely to be most useful for interpreting gene expression?
%
%      \item Methods:
%    
%    Method1:Direct visual inspection(e.g.heatmap)
%
%    Method2:hierarchical clustering
%    
%    Method3:Bayesian clustering
%         
%    Method4:K-means clustering
%         
%    Method5:Self-organizing maps (SOM)
%
%    etc. ...
%    \end{itemize}
%\end{frame}
