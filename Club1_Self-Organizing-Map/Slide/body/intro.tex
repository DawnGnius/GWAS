\section{Intruduction}
\begin{frame}
\sectionpage
\end{frame}

\begin{frame}
    \frametitle{Intruduction}
    \begin{itemize}
        \item Background:
        
        array technologies:monitor the expression pattern of genes simultaneously;

        \item Challenge:
        
        interpret massive data sets in many fields;
        
        e.g.object recognition by machine vision systems;

        \hspace{1.2em} phoneme detection in speech processing;

        \hspace{1.3em} bandwidth compression in telecommunications;

        \hspace{1.3em} etc.

        \item Essential problem:
        
        cluster points in multidimensional space;
    \end{itemize}
\end{frame}



\begin{frame}
    \frametitle{Microarray gene expression}
    \begin{figure}[h]
        \centering
        \includegraphics[width=0.7\textwidth]{./figs/w_fig3.png}
        \label{Microarray gene expression}
        \caption{Microarray gene expression}
    \end{figure}
\end{frame}

\begin{frame}
    \frametitle{clustering}
    \begin{figure}[h]
        \centering
        \includegraphics[width=0.7\textwidth]{./figs/w_fig2.png}
        \label{clustering}
        \caption{clustering}
    \end{figure}
\end{frame}


       


\begin{frame}
    \frametitle{Methods of clustering}
    \begin{itemize}
      \item Question:
   
    Which clustering techniques are likely to be most useful for interpreting gene expression?

      \item Methods:
    
    Method1:Direct visual inspection(e.g.heatmap)

    Method2:hierarchical clustering
    
    Method3:Bayesian clustering
         
    Method4:K-means clustering
         
    Method5:Self-organizing maps (SOM)

    etc.
    \end{itemize}
\end{frame}

\begin{frame}
    \frametitle{Method1:Direct visual inspection}
    \begin{itemize}
      \item Direct visual inspection(e.g.heatmap)

          Advantages: 

              best suited for instances in which the patterns of interest are clear in advance;

          Disadvantages:

              does not scale very well to larger data sets;

              less appropriate for discovering unexpected patterns;
    \end{itemize}
\end{frame}

\begin{frame}
    \frametitle{Method2:Hierarchical clustering}
    \begin{itemize}
      \item Hierarchical clustering

    Key points:

    data points: forced into a strict hierarchy of nested subsets;

    the closest pair of points: grouped and replaced by a single
    
    point representing their set average;

    the next closest pair of points: treated similarly;
    \end{itemize}
    
\end{frame}

\begin{frame}
    \frametitle{Method2:Hierarchical clustering(continiued)}
    \begin{itemize}
      \item Method2:Hierarchical clustering

    Advantage: 

    \hspace{0.6em}  best suited to situations of true hierarchical descent;

    Disadvantages:

    \hspace{0.6em}  computation;

    \hspace{0.6em}  local optimal;

    \hspace{0.6em}  cannot reflect the multiple distinct;

    \hspace{0.6em}  lack of robustness;

    \hspace{0.6em}  nonuniqueness;

    \hspace{0.6em}  inversion problems;
   \end{itemize}
\end{frame}

    

\begin{frame}
    \frametitle{Method2:Hierarchical clustering(continiued)}
    \begin{figure}[h]
        \centering
        \includegraphics[width=0.3\textwidth]{./figs/w_fig1.png}
        \label{Table1}
        \caption{Hierarchical clustering can have inversions.}
    \end{figure}
\end{frame}

    
\begin{frame}
    \frametitle{Method3:K-means clustering}
    \begin{itemize}   
     \item K-means clustering
     
       Advantages: 

        \hspace{0.6em} intuitive and easy to understand;

        \hspace{0.6em} relatively simple and efficient;

       Disadvantages:

       \hspace{0.6em} local optimum;

       \hspace{0.6em} define the "mean" of categorical data;

       \hspace{0.6em} specify K in advance;

       \hspace{0.6em} unable to handle noisy data and outliers;

       \hspace{0.6em} not suitable to discover clusters with non-convex shapes;
    \end{itemize}
\end{frame}

\begin{frame}
    \frametitle{Other methods}
    \begin{itemize}
     \item Method4:Bayesian clustering

    \hspace{0.9em} highly structured approach when a strong prior distribution 
    
    \hspace{0.9em} on the data is available; 

    \item Method5:Topic Self-organizing maps(SOM) 

    etc.
\end{itemize}
\end{frame}

% \begin{frame}
%     \frametitle{Precedure of LDPE}
%     \begin{figure}[h]
%         \centering
%         \includegraphics[width=1.0\textwidth]{./figs/Table1.png}
%         \label{Table1}
%         \caption{This is a figure}
%     \end{figure}
% \end{frame}


