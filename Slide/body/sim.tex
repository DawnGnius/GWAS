\section{Visualized Simulation}
\begin{frame}
\sectionpage
\end{frame}

\begin{frame}
    \frametitle{Model and Setting}
    \begin{itemize}
        \item Model
        
        \hspace{2em} $X \sim \frac{1}{K} \sum_{i=1}^K \mathcal{N}(\mu_i, I)$

        \item Setting

        \hspace{2em} Number of observations $N= 300$;

        \hspace{2em} Number of clusters $K = 6$;

        \hspace{2em} $\mu_i = (0,0), (4,10), (10,4), (15,15), (0,20), (20, 0)$.

        \item SOM
        
        \hspace{2em} Initialization with a rectangle grid covering the data, 

        \hspace{2em} Choosing \textbf{Batch size}, \textbf{Neighborhood} function $D$ and \textbf{learning rate} function $R$.
    \end{itemize}
\end{frame}

\begin{frame}
    \frametitle{Visualization}
    \vspace{-3em}
    \begin{figure}[!htp]
        \centering
        \animategraphics[width=1.0\textwidth,autoplay,controls]{2}{./figs/fig}{0}{35}
        \caption{2-D SOM Clustering Video}
    \end{figure}
    See: \href{http://home.ustc.edu.cn/~huihang/som_b.gif}{Online material} ``\url{http://c3v.cn/3iqVG}'' made by ourselves. 
\end{frame}

\begin{frame}
    \frametitle{Visualization Continue: Grid Initialization}
    \begin{figure}[!hp]
        \centering
        \includegraphics[width=1.0\textwidth]{./figs/fig0.png}
        \caption{Grid Initialization}
        \label{fig:grid}
    \end{figure}
    We generate a $3\times 3$ rectangle grid.
\end{frame}

% Stage 1
\begin{frame}
    \frametitle{Visualization Continue: Iteration stage 1}
    \begin{figure}[!hp]
        \centering
        \includegraphics[width=1.0\textwidth]{./figs/fig1.png}
        \caption{Initial stage of iteration}
    \end{figure}
    
    \textbf{Radius of Neighborhood} and \textbf{Learning Rate} are relatively \textbf{large}. 
\end{frame}

\begin{frame}
    \frametitle{Visualization Continue: Iteration stage 1}
    \begin{figure}[!hp]
        \centering
        \includegraphics[width=1.0\textwidth]{./figs/fig2.png}
        \caption{Initial stage of iteration (continue)}
    \end{figure}
    
    Nodes are \textbf{widely influenced} by inputs. 
\end{frame}

% Stage 2
\begin{frame}
    \frametitle{Visualization Continue: Iteration stage 2}
    \begin{figure}[!hp]
        \centering
        \includegraphics[width=1.0\textwidth]{./figs/fig10.png}
        \caption{Intermediate stage of iteration}
    \end{figure}
    
    \textbf{Radius of Neighborhood} decreases with time $t$. \textbf{Learning Rate} decreases with $t$ and $D$.
\end{frame}

\begin{frame}
    \frametitle{Visualization Continue: Iteration stage 2}
    \begin{figure}[!hp]
        \centering
        \includegraphics[width=1.0\textwidth]{./figs/fig11.png}
        \caption{Intermediate stage of iteration (continue)}
    \end{figure}
    
    Each input will influence the Node around them, \textbf{WTA}.
\end{frame}

% Stage 3
\begin{frame}
    \frametitle{Visualization Continue: Iteration stage 3}
    \begin{figure}[!hp]
        \centering
        \includegraphics[width=1.0\textwidth]{./figs/fig20.png}
        \caption{Last stage of iteration}
    \end{figure}
    
    \textbf{Radius of Neighborhood} drop to 0, and \textbf{Learning Rate} keep decreasing.
\end{frame}

\begin{frame}
    \frametitle{Visualization Continue: Iteration stage 3}
    \begin{figure}[!hp]
        \centering
        \includegraphics[width=1.0\textwidth]{./figs/fig30.png}
        \caption{Last stage of iteration (continue)}
    \end{figure}
    
    Each input will only \textbf{slightly} influence the most \textbf{nearest/simmilar} Node, \textbf{WTA}. 
\end{frame}

% overview
\begin{frame}
    \frametitle{Clustering Result}
    \begin{figure}[!hp]
        \centering
        \includegraphics[width=1.0\textwidth]{./figs/fig36.png}
        \caption{Result of SOM. There are 7 clusters with different color in left panel. 
        The right panel shows different patterns of different clusters. 
        Note that the order of the clusters in right panel is the same with the rectangle Node Grid in Fig~\ref{fig:grid}.}
    \end{figure}
\end{frame}

\begin{frame}
    \frametitle{UCI ML repository}
    \begin{table}[htbp]
        \centering
        \caption{$5$ Popular Clustering Data Sets 
        \footnote[1]{\url{http://archive.ics.uci.edu/ml/}}}
        \label{table1}
        \begin{tabular}{l|cclr}  
            \toprule
                        & Obser.& Attr.     & Area          & Categorical \footnote[2]{R: Real data , I: Integer data, C: Categorical data}  \\
            \midrule
            Iris        & $150$   & $4$         & Plant         & R       \\
            Wisconsin   & $198$   & $34$        & Breast cancer & R       \\
            Dermatology & $366$   & $33$        & Dermatology   & C, I    \\
            Pima        & $768$   & $8$         & Diabetes      & I, R    \\
            Abalone     & $4177$  & $8$         & Animal        & C, I, R \\
            \bottomrule
        \end{tabular}
    \end{table}
\end{frame}
\begin{frame}
    \frametitle{Other Experiment: UCI ML repository}    
    \begin{table}[htbp]
        \centering
        \caption{Comparison the Experimental results between three methodologies
        \footnote[1]{Boscarioli, Clodis , R. Villwock , and B. E. Soares . ``Comparing an Ant-Based Clustering Algorithm with Self-Organizing Maps and K-means.'' International Journal of Computer Science \& Network Security (2012).}}
        \label{table1}
        \begin{tabular}{l|ccccc}  
            \toprule
            &  Iris &  Wisconsin &  Dermatology & Pima & Abalone \\ 
            \midrule
            % Ant Colony & 0.787 & 1.864 & 3.802 & 4.600 & 29.599 \\
            % \midrule
            K-means & $0.846$ & $1.444$ & $1.502$ & $2.174$ & $1.034$ \\
            % \midrule
            SOM &  $0.749$&  $0.695$ & $0.839$&  $0.723$&  $0.887$\\
            \bottomrule
        \end{tabular}
    \end{table}

\end{frame}